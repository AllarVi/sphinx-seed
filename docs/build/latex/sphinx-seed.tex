%% Generated by Sphinx.
\def\sphinxdocclass{report}
\documentclass[letterpaper,10pt,english]{sphinxmanual}
\ifdefined\pdfpxdimen
   \let\sphinxpxdimen\pdfpxdimen\else\newdimen\sphinxpxdimen
\fi \sphinxpxdimen=.75bp\relax

\usepackage[utf8]{inputenc}
\ifdefined\DeclareUnicodeCharacter
 \ifdefined\DeclareUnicodeCharacterAsOptional
  \DeclareUnicodeCharacter{"00A0}{\nobreakspace}
  \DeclareUnicodeCharacter{"2500}{\sphinxunichar{2500}}
  \DeclareUnicodeCharacter{"2502}{\sphinxunichar{2502}}
  \DeclareUnicodeCharacter{"2514}{\sphinxunichar{2514}}
  \DeclareUnicodeCharacter{"251C}{\sphinxunichar{251C}}
  \DeclareUnicodeCharacter{"2572}{\textbackslash}
 \else
  \DeclareUnicodeCharacter{00A0}{\nobreakspace}
  \DeclareUnicodeCharacter{2500}{\sphinxunichar{2500}}
  \DeclareUnicodeCharacter{2502}{\sphinxunichar{2502}}
  \DeclareUnicodeCharacter{2514}{\sphinxunichar{2514}}
  \DeclareUnicodeCharacter{251C}{\sphinxunichar{251C}}
  \DeclareUnicodeCharacter{2572}{\textbackslash}
 \fi
\fi
\usepackage{cmap}
\usepackage[T1]{fontenc}
\usepackage{amsmath,amssymb,amstext}
\usepackage{babel}
\usepackage{times}
\usepackage[Bjarne]{fncychap}
\usepackage[dontkeepoldnames]{sphinx}

\usepackage{geometry}

% Include hyperref last.
\usepackage{hyperref}
% Fix anchor placement for figures with captions.
\usepackage{hypcap}% it must be loaded after hyperref.
% Set up styles of URL: it should be placed after hyperref.
\urlstyle{same}
\addto\captionsenglish{\renewcommand{\contentsname}{Contents:}}

\addto\captionsenglish{\renewcommand{\figurename}{Fig.}}
\addto\captionsenglish{\renewcommand{\tablename}{Table}}
\addto\captionsenglish{\renewcommand{\literalblockname}{Listing}}

\addto\captionsenglish{\renewcommand{\literalblockcontinuedname}{continued from previous page}}
\addto\captionsenglish{\renewcommand{\literalblockcontinuesname}{continues on next page}}

\addto\extrasenglish{\def\pageautorefname{page}}

\setcounter{tocdepth}{1}



\title{sphinx-seed Documentation}
\date{Oct 16, 2018}
\release{0.0.1}
\author{Allar}
\newcommand{\sphinxlogo}{\vbox{}}
\renewcommand{\releasename}{Release}
\makeindex

\begin{document}

\maketitle
\sphinxtableofcontents
\phantomsection\label{\detokenize{index::doc}}



\chapter{Näidis lehekülg}
\label{\detokenize{samplepage:simple-documentation}}\label{\detokenize{samplepage::doc}}\label{\detokenize{samplepage:naidis-lehekulg}}
Nii lihtne ja tore on tegelikult doci kirjutada. Isegi pikemaid tekste on mugav kirjutada.


\section{Ja see on alampealkiri}
\label{\detokenize{samplepage:ja-see-on-alampealkiri}}
Veidi juttu siia ka.


\section{Õpetus, kuidas kirjutada Markdowni Atomis}
\label{\detokenize{samplepage:opetus-kuidas-kirjutada-markdowni-atomis}}
Aitab seadistada Atomit, et oleks mugav Markdowni kirjutada
https://www.portent.com/blog/content-strategy/atom-markdown.htm


\section{Proovin teha siia tabeli (TODO: ei tööta pdf-is)}
\label{\detokenize{samplepage:proovin-teha-siia-tabeli-todo-ei-toota-pdf-is}}
Selleks on vaja teha paar lisasammu
\begin{enumerate}
\item {} 
Installi lisavidin

\end{enumerate}

\sphinxcode{pip install sphinx-markdown-tables}
\begin{enumerate}
\item {} 
Lisa see conf.py-sse

\end{enumerate}

\fvset{hllines={, ,}}%
\begin{sphinxVerbatim}[commandchars=\\\{\}]
\PYG{n}{extensions} \PYG{o}{=} \PYG{p}{[}
    \PYG{l+s+s1}{\PYGZsq{}}\PYG{l+s+s1}{sphinx\PYGZus{}markdown\PYGZus{}tables}\PYG{l+s+s1}{\PYGZsq{}}\PYG{p}{,}
\PYG{p}{]}
\end{sphinxVerbatim}




\chapter{Indices and tables}
\label{\detokenize{index:indices-and-tables}}\begin{itemize}
\item {} 
\DUrole{xref,std,std-ref}{genindex}

\item {} 
\DUrole{xref,std,std-ref}{modindex}

\item {} 
\DUrole{xref,std,std-ref}{search}

\end{itemize}



\renewcommand{\indexname}{Index}
\printindex
\end{document}